\documentclass[twocolumn,draft]{article}
\usepackage{savetrees}
% a more conservative option:
% \documentclass[]{article}
% \usepackage{fullpage}

\usepackage[utf8]{inputenc}
\usepackage[T1]{fontenc}
\usepackage{microtype}

\usepackage{amssymb}
\usepackage{amsfonts}
\usepackage{amsmath}
\usepackage{amsthm}

\usepackage{todonotes}
\usepackage{graphicx}
\usepackage{showlabels}

\usepackage[english]{babel}
\usepackage{blindtext}
% \Blindtext to use.

\RequirePackage[l2tabu, orthodox]{nag}
\usepackage[all,warning]{onlyamsmath}

\usepackage{hyperref}

% red and blue text for questions
\newcommand{\rt}[1]{\textcolor{red}{#1}}
\newcommand{\bt}[1]{\textcolor{blue}{#1}}



\title{MA 341, 4/7/17}
\author{John M. Lynch\footnote{Undergraduate ECE/Physics, NCSU, Raleigh, NC 27705. E-Mail: \texttt{jmlynch3@ncsu.edu}}}
\date{\today}

\begin{document}
  \maketitle
  
  \section*{Midterm Review}
  
  Target: systems of differential equations.
  
  Outline of topics (similar to last time):
  
  \begin{enumerate}
	  
  	\item \rt{Homogeneous systems}
		\begin{equation*}
			\frac{dx}{dt}= Ax
		\end{equation*}
		Solve for:
		\begin{enumerate}
			\item gen solution
			\item IVP
			\item fundamental matrix
			\item how many eigenvectors
		\end{enumerate}
		Won't have to worry about edge cases (e.g. insufficient number of eigenvectors)
		
	\item \rt{Non-Homogeneous Systems}
		\begin{equation*}
			\frac{dx}{dt} = Ax + \rho(t)
		\end{equation*}
		Using
		\begin{enumerate}
			\item Undetermined coefficients
			\item Variation of parameters
		\end{enumerate}
		Assumes you can handle homogeneous systems and that you can \emph{integrate by parts}
		
	\item \rt{Eigenvector Properties}
		\begin{enumerate}
			\item given eigenvalues $r_{1}, \ldots, r_{n}$, all distinct, multiplicity $1$,
				the corresponding eigenvectors are independent
			\item multiple eigenvalues: if sufficiently many eigenvectors exist, they can be
				selected as independent ones.
		\end{enumerate}
	
	\item 
  \end{enumerate}
  
  \section*{Examples}
  
  \begin{enumerate}
  	\item \bt{Inverting a matrix (Gauss-Jordan)}
		\begin{align*}
			\left[
			\begin{array}{ccc|ccc}
				1 & 1 & 1  & 1 & 0 & 0 \\
				1 & -1 & 2 & 0 & 1 & 0 \\
				1 & 1 & 4  & 0 & 0 & 1
			\end{array} \right] \\
			&= 
			\left[
			\begin{array}{ccc|ccc}
				1 & 1 & 1  & 1 & 0 & 0 \\
				0 & -2 & 1 & -1 & 1 & 0 \\
				1 & 0 & 1  & \frac{-1}{3} & 0 & 1
			\end{array} \right] \\
			&= 
			\left[
			\begin{array}{ccc|ccc}
				1 & 0 & 0  & 1 			  & \frac{1}{2} 	& \frac{-1}{2} \\
				0 & 1 & 0  & \frac{1}{3}  & \frac{-1}{2} 	& \frac{1}{6} \\
				0 & 0 & 1  & \frac{-1}{3} & 0 				& \frac{1}{3}
			\end{array} \right] \\
				A^{-1} &= 
				\begin{bmatrix}
					1 			  & \frac{1}{2} 	& \frac{-1}{2} \\
					\frac{1}{3}  & \frac{-1}{2} 	& \frac{1}{6} \\
					\frac{-1}{3} & 0 				& \frac{1}{3}
				\end{bmatrix}
		\end{align*}
	
	\item \bt{Multiple Eigenvalues}
		\begin{equation*}
			A = 
			\begin{bmatrix}
				1 & 2 & -1 \\
				0 & 1 & 0 \\
				0 & 1 & 1
			\end{bmatrix}
		\end{equation*}
		
		\begin{align*}
			A-rI &=
			\begin{vmatrix}
				1-r & 2 & -1 \\
				0 & 1-r & 0 \\
				0 & 1 & 1-r
			\end{vmatrix} \\
			&= (1-r)\left[(1-r)^{2}\right] \\
			&= (1-r)^{3} \\
			r &= 1 \tag*{multiplicity 3}
		\end{align*}

		Plug $r$ into the matrix
		
		\begin{align*}
			\begin{bmatrix}
				0 & 2 & -1 \\
				0 & 0 & 0 \\ 
				0 & 1 & 0
			\end{bmatrix}
		\end{align*}
		
		and extract equations:
		
		\begin{align*}
			2u_{2} - u_{3} &= 0 \\
			u_{2} &= 0 \\
			0 &= 0
		\end{align*}
		
		which is an insufficient number of eigenvectors.
		
		\begin{align*}
			u_{2} = u_{3} &= 0 \\
			0 &= 0 \\
			&\Rightarrow
			\begin{bmatrix}
				s \\ 0 \\ 0
			\end{bmatrix} \\
			&= s \begin{bmatrix}
				 	   1 \\ 0 \\ 0
				 \end{bmatrix}
		\end{align*}
	
	\item \bt{Multiple Eigenvalues (II)}
		\begin{align*}
			A &= \begin{bmatrix}
					-1 & 1 & 0 \\
					 1 & 2 & 1 \\
					 0 & 3 & 1
				 \end{bmatrix} \\
			\text{det}(A-rI) &= \begin{vmatrix}
									-1-r & 1 & 0 \\
									 1 & 2-r & 1 \\
									 0 & 3 & 1-r
								\end{vmatrix} \\
				 &= -(1+r)\left[(2-r)(-1-r)-3\right] \\
				 &= -1\left[(-1-r)-1-0\right] \\
				 &= (1+r)\left[r^{2}-r-2-3\right] + (1+r) \\
				 &= -(1+r)(r-3)(r+2) \\
				 r_{1} &= -1 \\
				 r_{2} &= -2 \\
				 r_{3} &= 3
		\end{align*}
		
		Plug each into A-rI:
		
		\begin{align*}
			   &\boxed{r = -1}: \\
			A-rI &= \begin{bmatrix}
						0 & 1 & 0 \\
						1 & 3 & 1 \\
						0 & 3 & 0
					\end{bmatrix} \\
					   u_{2} &= 0 \\
			   u_{1} + u_{3} &= 0 \\
			   u &= \begin{bmatrix}
			   			-s \\ 0 \\ s
			   		\end{bmatrix} \\
			  s &= \begin{bmatrix}
			  			-1 \\ 0 \\ 1
			  	   \end{bmatrix} \\
			&\boxed{r = -2}: \\
	   A - rI &= \begin{bmatrix}
	   				1 & 1 & 0 \\
					1 & 4 & 1 \\
					0 & 3 & 1
	   			 \end{bmatrix} \\
				 u_{1} + u_{2} &= 0 \\
				 u_{1} + 4u_{2} + u_{3} &= 0 \\
				 3u_{2} + u_{3} &= 0 \\
				 u &= s \begin{bmatrix}
				 			-1 \\ 1 \\ -3
				 		\end{bmatrix} \\
			&\boxed{r = 3}: \\
			4u_{1} + u_{2} &= 0 \\
			u_{1} - u_{2} + u_{3} = 0 \\
			3u_{2} - 4u_{3} = 0 \\
			% new eqns
			u_{1} - u_{2} + u_{3} &= 0 \\
			-3u_{2} + 4u_{3} &= 0 \\
			u_{2} &= 4s \\
			u_{3} &= 3s \\
			u_{1}-4s+3s &= 0 \\
			u_{1} &= s \\
			u &= \begin{bmatrix}
					s \\ 4s \\ 3s
				\end{bmatrix} \\
			  &= s
			  \begin{bmatrix}
			  	 	1 \\ 4 \\ 3
			  \end{bmatrix}
		\end{align*}
		
		The system of algebraic equations will reduce to a single equation.
		
		If multiple:
		
		\begin{align*}
			u_{1} -u_{2} + u_{3} &= 0 \\
			u_{2} &= s_{1} \\
			u_{3} &= s_{2} \\
			u_{1} = s_{1}-s_{2} \\
			u &= \begin{bmatrix}
					s_{1}-s_{2} \\
					s_{1} \\ 
					s_{2}
				\end{bmatrix} \\
				% split into (s1, s1, 0) + (-s2, 0 ,s2)
		\end{align*}
  \end{enumerate}
  
 
\end{document}